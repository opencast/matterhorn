\documentclass{sig-alternate}

\begin{document}
\conferenceinfo{ACM MM}{2011 Scottsdale, Arizona, USA}
\title{Opencast Matterhorn}
\numberofauthors{8} % total number including ones at end

\author{
%
\alignauthor
		Christopher A. Brooks \titlenote{Corresponding author.}\\
       \affaddr{University of Saskatchewan}\\
       \affaddr{Saskatoon, SK, Canada}\\
       \email{cab938@mail.usask.ca}
%
\alignauthor
		Markus Ketterl\\
       \affaddr{Universit\"{a}t Osnabr\"{u}ck}\\
       \affaddr{Osnabr\"{u}ck, Germany}\\
       \email{mketterl@uni-osnabrueck.de}
%
\alignauthor
		Adam Hochman\\
       \affaddr{University of California at Berkeley}\\
       \affaddr{Berkeley, CA, USA}\\
       \email{adam@media.berkeley.edu}
\and
%
\alignauthor
		Josh Holtzman\\
       \affaddr{University of California at Berkeley}\\
       \affaddr{Berkeley, CA, USA}\\
       \email{jholtzman@berkeley.edu}
%
\alignauthor
		Tobias Wunden\\
       \affaddr{Eidgen\"{o}ssische Technische Hochschule Z\"{u}rich}\\
       \affaddr{Zurich, Switzerland}\\
       \email{tobias.wunden@id.ethz.ch}
%
\alignauthor
		Judy Stern\\
       \affaddr{University of California at Berkeley}\\
       \affaddr{Berkeley, CA, USA}\\
       \email{judy@media.berkeley.edu}
}


\maketitle
\begin{abstract}
\end{abstract}

\keywords{Human, Social, and Educational Aspects of Multimedia, Multimedia Systems and Middleware}

\section{Introduction}
The Opencast community is a collaboration of higher education institutions, commercial partners, and individuals working together to explore, create, and document best practices and technologies with respect to rich media in higher education.  The principal activity this community has undergone is the development of the Matterhorn product: an open source and enterprise strength platform for media acquisition, processing, distribution, and playback.

In 2009 a group of thirteen higher education institutions came together and formed a team of designers, developers, and researchers charged with creating a open source solution for lecture capture in the enterprise.  Consisting of members from computer science research groups as well as institutional education and technology infrastructure support groups, the aim of this team was to take lessons learned from deployed home grown systems and build a scalable, accessible, open solution for higher education.

Supported by a Mellon and Hewlett Foundation grant totalling \$1.5M, and institutional matching funds of \$2M, development began in MONTH/YEAR and is now maintained by over X committers from institutions and corporations from around the world.  For more information on Opencast Matterhorn readers are invited to visit www.opencastproject.org.

\section{Architecture}
\section{Production System}
\section{Research Platform}
One of the powerful aspects of a common platform for lecture capture in higher education is the ability to share not only research results but research innovations.  Several institutions are actively developing, deploying, and validating systems based on Matterhorn, and the ease at which these institutions can share those results back with the broader community, allows for greater impact while reducing the costs of applying research in practice.
A sample of some of the research endevours using Matterhorn that we are familiar with includes:
\begin{itemize}
\item Data mining student traces for cohort rec.
\item Scene detection using SML
\item Speech rec. ETH
\item Footprints, Osna
\item Opentrack who?
\end{itemize}

\section{Conclusions}

\section{Introduction}


\section{Acknowledgments}
Hewlett and Mellon

\bibliographystyle{abbrv}
\bibliography{sigproc}  % sigproc.bib is the name of the Bibliography in this case

\appendix
\section{Additional Authors}
The cover page lists authors who are members of the product owner team and executed the vision for Matterhorn 1.0.  Additional authors of this work include our numerous committers and designers (listed here alphabetically): a ({\texttt{a@a.org}}) and b ({\texttt{b@b.net}}).

\section{References}

\end{document}
