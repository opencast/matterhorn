\documentclass{sig-alternate}

\begin{document}
\conferenceinfo{ACM MM}{2011 Scottsdale, Arizona, USA}
\title{Opencast Matterhorn}
\numberofauthors{8} % total number including ones at end

\author{
%
\alignauthor
		Christopher A. Brooks \titlenote{Corresponding authors.  Other authors listed alphabetically.  Corresponding authors are Ph.D. students at their institutions.}\\
       \affaddr{University of Saskatchewan}\\
       \affaddr{Saskatoon, SK, Canada}\\
       \email{cab938@mail.usask.ca}
%
\alignauthor
		Markus Ketterl \titlenote{}\\
       \affaddr{Universit\"{a}t Osnabr\"{u}ck}\\
       \affaddr{Osnabr\"{u}ck, Germany}\\
       \email{mketterl@uni-osnabrueck.de}
%
\alignauthor
		Adam Hochman\\
       \affaddr{University of California at Berkeley}\\
       \affaddr{Berkeley, CA, USA}\\
       \email{adam@media.berkeley.edu}
\and
%
\alignauthor
		Josh Holtzman\\
       \affaddr{University of California at Berkeley}\\
       \affaddr{Berkeley, CA, USA}\\
       \email{jholtzman@berkeley.edu}
%
\alignauthor
		Judy Stern\\
       \affaddr{University of California at Berkeley}\\
       \affaddr{Berkeley, CA, USA}\\
       \email{judy@media.berkeley.edu}
%
\alignauthor
		Tobias Wunden\\
       \affaddr{Eidgen\"{o}ssische Technische Hochschule Z\"{u}rich}\\
       \affaddr{Zurich, Switzerland}\\
       \email{tobias.wunden@id.ethz.ch}
}


\maketitle
\begin{abstract}
%todo: chris
\end{abstract}

\keywords{Human, Social, and Educational Aspects of Multimedia, Multimedia Systems and Middleware}

\section{Introduction}
The Opencast community is a collaboration of higher education institutions, commercial partners, and individuals working together to explore, create, and document best practices and technologies with respect to rich media in higher education.  The principal activity this community has undergone is the development of the Matterhorn product: an open source and enterprise strength platform for media acquisition, processing, distribution and playback for lecture capture.

In 2009 a group of thirteen higher education institutions formed a team of designers, developers, and researchers charged with creating a reliable open source solution for lecture capture and content usage.  Consisting of members from computer science research groups as well as institutional education and technology infrastructure support groups, the aim of this team was to take lessons learned from deployed home grown systems and build a scalable, accessible, open solution for higher education.  Notably, the system was intended to be both a platform for research in multimedia, pedagogy, and education, as well as a production system intended to scale to future needs, federations of institutions and tens of thousands of students accessing and using the content.

Supported by a Mellon and Hewlett Foundation grant totalling \$1.5M, and institutional matching funds of \$2M, development has continued for two years and is maintained by over 17 committers from institutions and corporations from around the world.  The Opencast community has over 650 members with more than 200 of them interested in the Matterhorn build project\footnote{As there are no membership fees or official membership process these numbers reflect subscriptions to our public mailing lists.}. More information about Opencast Matterhorn can be found in \cite{DBLP:journals/itse/KetterlSH10} and by visiting the projects homepage at www.opencastproject.org. 

\section{Architecture \& System}
Matterhorn is made up of three subsystems corresponding to content acquisition, processing, and playback.  In addition to these distinct systems, a variety of user interfaces and API's for administration and choreographing these subsystems are provided.  

Technology wise, Matterhorn is built as a J2EE OSGI system, allowing for services to be swapped out dynamically at runtime as needed.  API's are provided as REST endpoints, encouraging reuse outside of the Java domain.  There is a high reliance of Matterhorn on other open source solutions (\emph{gstreamer}, \emph{ffmpeg}, and \emph{apache felix} to name a few of the more significant frameworks used), and all contributed code is licensed under the OSI approved Educational Community License (ECL) 2.0.  While no proprietary code is part of Matterhorn, several proprietary hardware and software solutions have been integration by interested parties until now.

\subsection{Content Acquisition}
The principal method of content acquisition in Matterhorn is the \emph{capture agent}; a hardware device connected to classroom audio/video technology.  The reference platform for this agent is based on commodity laptop-style x86 hardware and works with a number of different video capture devices including NTSC/PAL-based cards (such as the hauppage and bt787 chipsets), usb webcams, and specialized VGA capture devices (such as the epiphan line of products).  A fully capable agent with both VGA (5 FPS) capture and NTSC/PAL capture (30/25 FPS) costs roughly \$1,500, and several vendors are beginning to offer embedded or value-added solutions to integrate with Matterhorn capture at a similar price point.

The agent polls a central service which provide iCalendar encoded schedules.  Fault tolerance is provided through caching and retry timeouts, allowing capture agents to be deployed in environments will intermittent networking such as wireless or mobile cart situations.  Schedules identify both the metadata associated with a capture (e.g. course call number, instructor name, etc.) as well as the devices to capture from.  Underlying the scheduling system the agent uses the \emph{gstreamer} framework to acquire, transform, and encode video.  The system allows for multiple specialized encoding pipelines to be run at once, and initial support for transforming video to still frames for confidence monitoring during capture is available.

Once captured, video and other artefacts are aggregated and packaged into an extensible structure called a \emph{MediaPackage} and ingested into the Matterhorn workflow system.

\subsection{Video Processing}
The core of Matterhorn is the video processing workflow system.  This system uses a service oriented architecture to dispatch jobs to worker nodes which may be specialized for particular services or provide a variety of different general purpose services.  Workflows are linear compositions of these services (which may be synchronous or asynchronous) meant to be run in parallel depending upon the infrastructure topology.  In this way Matterhorn can be scaled with more computing resources both to accommodate new services as well as to accommodate an increased load or responsiveness.

Content is ingested into the system in one of three ways; through REST APIs as described previously, through an ``inbox'' metaphor to accommodate legacy applications (e.g. those that may use file share copying, FTP, etc.), or through end-user web interfaces.  Once ingested, a workflow is started by the ingestion system which automatically distributes jobs across platform depending on worker loads.

Services included in the Matterhorn 1.1 distribution include video encoding (using \emph{ffmpeg} or \emph{gstreamer} as desired), metadata generation, scene detection, preview image generation, trimming and captioning (requiring user input from a web-based tools), and text analysis (using \emph{tesseract}).  A command line operation handler is also included allowing for end-user integration of other tools as services in a declarative manner.

Distribution of content to is also implemented using services.  Matterhorn publishing services for both progressive download and the open source \emph{red5} streaming server are included, as well as a local search index service using \emph{solr}.  Services for distribution to other popular portals such as YouTube.edu and iTunesU are under development.

\subsection{Media Playback}
%todo: markus
%-searching and rss as a "portal" make integration easy
%-a11y
%-flex and the open frameworks we are using for media
%-advanced features but don't touch on footprints, this comes later.

\section{Production System}
Matterhorn has been used at institutions across the globe as a lecture capture solution for thousands of students.  The open source nature of the product makes measurement of adoption difficult, however both corresponding authors of this document have overseen widespread local deployments of Matterhorn and offer a glimpse here of the effect it has had at their institution.  More user testimonials can be found on the Opencast Matterhorn project website.

\subsection{University of Saskatchewan}
The University of Saskatchewan piloted Matterhorn 1.0 capture agents in 11 classrooms between September 2010 and April 2011.  Custom Matterhorn workflows were created to connect to a legacy media player which had already been integrated into campus infrastructure.

Over this time period 944 events were captured with the majority of them being regular undergraduate lectures (between 1 hour and 4 hours long) along with some special events and meetings.  A breadth first approach was taken to lecture capture deployment, and a variety of disciplines were recorded including the natural sciences, social sciences, humanities, and professional colleges.

During this two term deployment, 2,639 unique students watched over 1,543 days worth of video, an average of roughly 41 hours per student.  Students were restricted to viewing videos from their enrolled classes only, though many students were in multiple classes.

In addition to this, several students at the university have been involved in the customization of Matterhorn through volunteering, summer research positions, professional internships, and graduate theses.  The University of Saskatchewan was a mentoring institution in the Undergraduate Capstone Open Source projects (UCOSP)\footnote{See http://http://ucosp.ca/} where five senior undergraduate students from institutions across the country worked on various aspects of Matterhorn for completion of their bachelors degree.

\subsection{Universit\"{a}t Osnabr\"{u}ck}
%todo: markus
%worth mentioning your software engineering class too

\section{Research Platform}
One of the powerful aspects of a common platform for lecture capture in higher education is the ability to share not only research results but research innovations.  Several institutions are actively developing, deploying, and validating systems based on Matterhorn, and the ease at which these institutions can share those results back with the broader community, allows for greater impact while reducing the costs of applying research in practice.
A sample of some of the research endevours using Matterhorn that we are familiar with includes:
\begin{itemize}
\item \textbf{Student interaction modelling through clustering.}   Using Matterhorn recorded lectures, clusters of students who share similar viewing patterns were discovered in \cite{Brooks2010}.  Tying these clusters to pedagogical goals and educational strategies can allow for content recommendation or early warning of student problems.
\item \textbf{Scene detection using Supervised Machine Learning.}  The Matterhorn scene detection service was analysed in \cite{Johnston2011} against several other scene detection techniques.  The workflow operation handler features of Matterhorn make both multiple detection techniques as well as supervised machine learning techniques feasible in large scale implementations if computational resources are available.
%todo: markus
%\item Speech rec. ETH (maybe we should leave this out?)  Its ok if there is no cite, just a paragraph would be fine.
%\item Footprints, Osna (anything new from virtpresenter?
%\item Opentrack who? osna?  eth?  both?
\end{itemize}

\section{Conclusions}
%todo: chris

\section{Acknowledgments}
We would like to thank the Hewlett and Mellon foundations for their critical support that allowed us to coordinate, plan, execute, and sustain the vision we have for media in higher education.

In addition to the many committers who made this work possible, we would like to thank the following committers emeritus and contributors who played a pivotal role in bringing the product to fruition (in alphabetical order): Stefan Altevogt, Allison Bloodworth, Johannes Emden, Mara Hancock, Jamie Hodge, Andre Klassen, Bostjan Pajntar, Olaf Shulte, and Aaron Zeckoski.

\bibliographystyle{abbrv}
\bibliography{sigproc}  % sigproc.bib is the name of the Bibliography in this case

\section{Additional Authors}
The cover page lists authors who are members of the product owner team and executed the vision for Matterhorn 1.0.  Additional authors of this work include our numerous committers and designers (listed here alphabetically): Kristofor Amundson, Greg Logan, Kenneth Lui, Adam McKenzie, Denis Meyer, Markus Moormann, Matjaz Rihtar, Ruediger Rolf, Nejc Skofic, Micah Sutton, Rub\'{e}n P\'{e}rez V\'{a}zquez, and Benjamin Wulff.  Full citation of this work should acknowledge these persons as extended authors.

\end{document}
